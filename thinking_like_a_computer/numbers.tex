\documentclass{article}
\usepackage{amsmath}
\usepackage{fullpage}
\usepackage{tabularx}

\title{Numbers}

\newcolumntype{C}{>{\centering\arraybackslash}X}

\begin{document}
\maketitle

\section{Overview}

The standard decimal numeral system that we all know and love is great for humans because having each digit represent some multiple of a power of $10$ is very convenient (and easy).
This is called a \textit{base-10} system.
This lesson will instead teach you about the binary (\textit{base-2}) and hexadecimal (\textit{base-16}) numeral systems which two other common numeral systems in the world of programming.
Why is that you might ask?
Well as you know, everything on a computer is represented as a $0$ or a $1$. 
Those are two numbers which naturally leads to a \textit{base-2} system.
However, writing everything as $0$s and $1$s sucks for humans; so we make it simpler for ourselves by writing numbers in \textit{base-16} format so we can easily figure out how the number was originally written in \textit{base-2} due to the fact the $16$ is $2^4$.
Note that I am not saying that you will never see \textit{base-10} while programming; you will still spend most of your time writing code using numbers written in \textit{base-10}; however, having an understanding of the \textit{base-2} and \textit{base-16} numeral systems will help your understanding of how a computer functions.

\section{Decimal Numbers}

Let us go back to the basics and re-examine what you know about \textit{base-10} numbers.
What does it actually mean for a number to be written as
\begin{equation}
1, 234, 567
\end{equation}
?
We know almost instinctively at this point that this number is one million, two-hundred thirty-four thousand, and five-hundred sixty-seven.
Why?
Because $7$ is in the ones place, $6$ is in the tens place, $5$ is in the hundredths place, etc.
In other words, we computed the number in our head as
\begin{align}
7 \times & 10^0 + \\
6 \times & 10^1 + \\
5 \times & 10^2 + \\
4 \times & 10^3 + \\
3 \times & 10^4 + \\
2 \times & 10^5 + \\
1 \times & 10^6 = 1,234,567
\end{align}
.
The important thing to note is that each place increases the exponent of the \textit{base-n} (in this case $10$) number by $1$.
It just so happens that the way we write and know numbers is identical to the decimal number system so this is all second nature to us.

\section{Binary Numbers}

So the only differences between decimal (\textit{base-10}) and binary numbers (\textit{base-2}) is that instead of raising $10$ to the power of the digit you raise $2$ to the power of the digit and that you will only use the numbers $0$ and $1$ to write out your number instead of using the numbers $0$ through $9$.

\section{Hexadecimal Numbers}

At this point, you should be able to guess that hexadecimal numbers (\textit{base-16}) numbers raise $16$ to the power of the digit instead of instead of $10$ or $2$.
However, instead of trying to cram the numbers $10$, $11$, $12$, $13$, $14$, and $15$ into a single digit, we use the letters \textbf{A}, \textbf{B}, \textbf{C}, \textbf{D}, \textbf{E}, and \textbf{F} instead.

\begin{table}
\centering
\caption{Hexadecimal digits.}
\begin{tabularx}{0.5\textwidth}{ |C|C| }
\hline
\textbf{Hexadecimal Digit} & \textbf{Decimal Equivalent} \\
A & $10$ \\
B & $11$ \\
C & $12$ \\
D & $13$ \\
E & $14$ \\
F & $15$ \\
\hline
\end{tabularx}
\end{table}

\end{document}
