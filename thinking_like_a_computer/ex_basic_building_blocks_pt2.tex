\documentclass{article}
\usepackage{amsmath}
\usepackage{fullpage}
\usepackage{listings}
\usepackage{exercise}

\lstset{
numbers=left, 
numberstyle=\small, 
numbersep=8pt, 
frame = single, 
framexleftmargin=15pt,
basicstyle = \ttfamily,
columns=fullflexible
}

\title{Basic Building Blocks: Exercises, Part 2}

\begin{document}
\maketitle

\begin{Exercise}

    Point out \textbf{all} the errors in the following piece of code.

    \begin{lstlisting}
        var x = 5
        var y = y + x
        var count = 0
        LOOP x < 10 {
            LOOP y {
                IF y % 2 == 0 {
                    count = count + 1
                }
                y = y + 1
            }
            x + 2
        }
    \end{lstlisting}

\end{Exercise}

\begin{Exercise}
    \begin{lstlisting}
        FUNCTION isEven(var x) {
            YOUR CODE HERE
        }
    \end{lstlisting}

    Replace \lstinline{YOUR CODE HERE} to check if the input variable \lstinline{x} is even or not.
    If it is even, return \lstinline{1} and if it is odd, return \lstinline{0}.
\end{Exercise}

\begin{Exercise}

    Assume you are given the following loop
    \begin{lstlisting}
        LOOP i < x {
            i = i + y
        }
    \end{lstlisting}

    How many times does the loop run if
    \begin{enumerate}
    \item $x = 10$ and $y = 1$
    \item $x = 11$ and $y = 2$
    \item $x = 523$ and $y = 1$
    \item $x = 12$ and $y = 4$
    \end{enumerate}
\end{Exercise}

\begin{Exercise}

    Assume you are given the following loop
    \begin{lstlisting}
        LOOP i <= x {
            i = i + y
        }
    \end{lstlisting}

    How many times does the loop run if
    \begin{enumerate}
    \item $x = 10$ and $y = 1$
    \item $x = 11$ and $y = 2$
    \item $x = 523$ and $y = 1$
    \item $x = 12$ and $y = 4$
    \end{enumerate}
\end{Exercise}

\begin{Exercise}

    Describe in words what this function does.

    \begin{lstlisting}
        FUNCTION someFunction(var a, var b) {
            IF a < b {
                RETURN b
            }
            RETURN a
        }
    \end{lstlisting}

\end{Exercise}

\begin{Exercise}

    Describe in words what this function does.

    \begin{lstlisting}
        FUNCTION someFunction(var a, var b) {
            var cur = a
            IF cur % 2 != 0 {
                cur = cur + 1
            }

            var count = 0
            LOOP cur <= b {
                IF cur % 3 == 0 {
                    count = count + 1
                }
                cur = cur + 2
            }

            return count
        }
    \end{lstlisting}

\end{Exercise}

\begin{Exercise}

    Assume you are given a function \lstinline{print} that takes in some variable \lstinline{x} and ``prints'' the result on your computer somehow (do not implement this).
    In otherwords, its \textbf{function signature} is
    \begin{lstlisting}
        FUNCTION print(var x)
    \end{lstlisting}
    Thus, to print the number $1$ to your screen you would write \lstinline{print(1)}.
    This function does not return any variables.
    Furthermore, assume you are given a function \lstinline{rand} that takes no variables as input and returns a random non-negative integer ($0$, $1$, $2$, ...).
    Its function signature is 
    \begin{lstlisting}
        FUNCTION rand()
    \end{lstlisting}
    and returns a single number.

    In this exercise, ``print'' all the odd numbers between \lstinline{a} and \lstinline{b} (inclusive).
    Finally, ``print'' the number of even numbers between \lstinline{a} and \lstinline{b} (inclusive).
    \begin{lstlisting}
        var a = rand()
        var b = rand()
    \end{lstlisting}

    Do not forget that \lstinline{a} can be greater than, less than, or equal to \lstinline{b}.

\end{Exercise}

\begin{Exercise}
    Assume once again, that you have access to the \lstinline{print} function from the previous exercise.
    Imagine you have a \lstinline{2x2} matrix (a matrix with $2$ rows and $2$ columns): 
    \begin{align*}
        \begin{bmatrix}
            0 & 1 \\
            2 & 3
        \end{bmatrix}
    \end{align*}
    The first row is row $0$, the second row is row $1$.
    Similarly, the first column is column $0$, the second column is column $1$.
    Assume that every element in the matrix can be computed as
    \begin{align*}
        n_c \times r + c
    \end{align*}
    where $n_c$ is the number of columns in the matrix, $r$ is the row of the current element, and $c$ is the column of the current element.
    Thus, in the \lstinline{2x2} case, for the top right element (row $0$, column $1$), you get
    \begin{align*}
        2 \times 0 + 1 = 1
    \end{align*}

    In this exercise, you will implement a function that will print all the numbers above the diagonal (i.e. $c > r$) for a matrix of arbitrary size.
    \begin{lstlisting}
        FUNCTION printNumbersAboveDiagonal(var numRows, var numColumns) {
            YOUR CODE HERE
        }
    \end{lstlisting}
    Here, \lstinline{var numRows} and \lstinline{var numColumns} tell you how many rows and columns your matrix has.
    In the case where your matrix has 2 rows and 2 columns, your function should only print $1$.
    In the case where your matrix has 3 rows and 3 columns, your function should print $1$, $2$, and $5$.

\end{Exercise}


\end{document}
