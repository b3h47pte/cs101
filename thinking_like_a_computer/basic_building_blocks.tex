\documentclass{article}
\usepackage{amsmath}
\usepackage{fullpage}
\usepackage{listings}
\usepackage{exercise}
\usepackage{hyperref}

\lstset{
numbers=left, 
numberstyle=\small, 
numbersep=8pt, 
frame = single, 
framexleftmargin=15pt}

\title{Basic Building Blocks}

\begin{document}
\maketitle

\section{Overview and Preface}

This lesson will teach you the basic building blocks of programming which are
\begin{itemize}
\item Variables,
\item Loops,
\item Conditionals,
\item Functions,
\item and Structs,
\end{itemize}
Most of computer science and software engineering is about how to effectively use the above concepts to accomplish some goal whether it be making a website like Google or running computer simulations for spaceship engines.
Note that there is a distinction between the disciplines of \textbf{computer science} and \textbf{software engineering}.
In very simplified terms, \textbf{computer science} is about finding new ways to solve some problem using a computer and \textbf{software engineering} is about finding the best way to tell the computer to solve the problem \textit{and} allowing you and your coworkers/fellow students to understand the solution weeks/months/years later.
In fact, there are many computer scientists who are horrendous software engineers and many software engineers who are terrible computer scientists \footnote{I will note that this is unideal and to be truly effective, one should know both.}.
However, they both fields share the common toolset of ``programming''.
My goal is to give you a solid foundation in programming so that you can excel in both should you choose.

Now, a common misconception about programming is that it requires one to excel in math.
However, that is simply not the case.
In fact, learning to become an effective programmer is probably more similar to learning a new language.
You certainly do not need to know multivariate calculus and partial differential equations to excel as a software engineer who creates websites and mobile applications for a living.
On the flip side, the more in depth you go in computer science, the more you will see the similarities between computer science and applied math so a strong foundation in math is definitely encouraged.

This lesson will focus on slowly introducing you to the building blocks of programming without burdening you with the harsh syntax requirements of real programming languages.
We will use an ad hoc pseudo-language to present these concepts; however, most programming languages will an equivalent so these skills should be easily transferable.
Note that I will use the word \textit{code} and \textit{program} interchangeably and I will refer to the \textit{code} written using our pseudo-language as \textit{pseudocode}.

\section{Variables}

Programming variables are very similar to the variables you learned about in Algebra.
Using your knowledge of Algebra, one can determine from Equation \ref{eq:algebra} that $y = 4$ and that $x$ and $y$ are the \textit{variables}.
\begin{equation}
\begin{aligned}
x & = 1 \\
y & = x + 3
\end{aligned}
\label{eq:algebra}
\end{equation}
Let us rewrite Equation \ref{eq:algebra} using our pseudo-language as seen in Listing \ref{code:algebra_pseudo}.
\begin{lstlisting}[caption={The equivalent of Equation \ref{eq:algebra} in our pseudo-language.}, label={code:algebra_pseudo}]
var x = 1
var y = x + 3
\end{lstlisting}
Here, we use the \lstinline{var} keyword to indicate that we are introducing a \textbf{new} \textit{variable} with the name \lstinline{x}.
The \lstinline{= 1} means the same thing as in Algebra; we are simply saying that we are assigning the \textit{variable} \lstinline{x} the value of $1$.
There is, however, one crucial difference between Algebra and programming.
\begin{equation}
\begin{aligned}
x & = 1 \\
y & = x + 3 \\
x & = x + 5
\end{aligned}
\label{eq:algebra_invalid}
\end{equation}
In math, Equation \ref{eq:algebra_invalid} makes no sense because
\begin{enumerate}
\item How can $x = 1$ while simultaneously be $x = x + 5$.
\item How can $x$ be defined using itself?
\end{enumerate}
However, the pseudocode shown in Listing \ref{code:algebra_invalid_pseudo} is perfectly valid.
\begin{lstlisting}[caption={The equivalent of Equation \ref{eq:algebra_invalid} in our pseudo-language.}, label={code:algebra_invalid_pseudo}]
var x = 1
var y = x + 3
x = x + 5
\end{lstlisting}
The first difference between code and Algebra is the concept of \textit{time}.
While Equation \ref{eq:algebra_invalid} is a singular statement of truth, Listing \ref{code:algebra_invalid_pseudo} are three separate statements that modify what is \textit{currently} true.
Thus, the proper way to read Listing \ref{code:algebra_invalid_pseudo} is in three separate steps and \textbf{each line of code gets executed in the order that they appear}.
\begin{enumerate}
\item After Line 1 executes, there is a variable \lstinline{x} with a value of $1$ and \textbf{no other variables exist}.
\item After Line 2 executes, \lstinline{x} still has a value of $1$ and there is now another variable \lstinline{y} with a value of $4$.
\item After Line 3 executes, \lstinline{x} has a value of $6$ and \lstinline{y} is unchanged.
\end{enumerate}
By convention, the right hand side of the equal sign is executed first which makes Line 3 perfectly valid.
The alternative way of writing that line is shown in Listing \ref{code:explicit_pseudo}; however, that requires two lines to write and creates a new variable and is thus less desirable.
\begin{lstlisting}[caption={A more explicit $x = x + 5$.}, label={code:explicit_pseudo}]
var z = x + 5
x = z
\end{lstlisting}
The true math equivalent of Listing \ref{code:algebra_invalid_pseudo} is thus shown in Equation \ref{eq:algebra_invalid_fix}.
\begin{equation}
\begin{aligned}
x_0 & = 1 \\
y & = x_0 + 3 \\
x & = x_0 + 5
\end{aligned}
\label{eq:algebra_invalid_fix}
\end{equation}
Finally, it is important to note that you can only set a variable equal to values that exist at that current moment in time.
Thus, writing \lstinline{var x = y + 2} in line 1 of Listing \ref{code:algebra_invalid_pseudo} would be invalid.
Similarly, writing \lstinline{var x = x + 2} in line 1 of Listing \ref{code:algebra_invalid_pseudo} would also be invalid.
Remember, the right hand side of the equal sign is executed first and at that point, \lstinline{var x} has not been executed to actually create the variable \lstinline{x}.

\section{Loops}

Sometimes one has the need to do something multiple times.
Take Listing \ref{code:loop_pseudo_unrolled} for example.
While it is somewhat contrived as we could just set \lstinline{var x = 6} to begin with, let us roll with it.
\begin{lstlisting}[caption={Unless you know what are you doing (you don't), don't do this.}, label={code:loop_pseudo_unrolled}]
var x = 1
x = x + 1
x = x + 1
x = x + 1
x = x + 1
x = x + 1
\end{lstlisting}
Effectively, this code is saying we want to increment \lstinline{x} (\lstinline{x = x + 1}) $5$ times.
Thus instead of writing lines $2$ through $6$ and duplicating code (generally bad!), we could write a loop instead as seen in Listing \ref{code:loop_pseudo}.
\begin{lstlisting}[caption={Loops are cool.}, label={code:loop_pseudo}]
var x = 1
var i = 0
LOOP i < 5 {
    x = x + 1
    i = i + 1
}
\end{lstlisting}
Here, the statement \lstinline|LOOP i < 5 { ... } | means that we want to \textbf{loop} (repeat) everything between the curly braces (\lstinline|{ ... }|) while the \textit{terminating condition} \lstinline{i < 5} is true.
Thus, Listing \ref{code:loop_pseudo} can be interpreted as
\begin{enumerate}
\item Create a new variable \lstinline{x} and set it to $1$.
\item Create a new variable \lstinline{i} and set it to $0$. \footnote{By convention, \textit{most} programming languages start counting at $0$, get used to it!}
\item Evaluate \lstinline{i < 5} ($i = 0$) to be \lstinline{true}.
\item Set $x = 2$.
\item Set $i = 1$.
\item Evaluate \lstinline{i < 5} ($i = 1$) to be \lstinline{true}.
\item Set $x = 3$.
\item Set $i = 2$.
\item Evaluate \lstinline{i < 5} ($i = 2$) to be \lstinline{true}.
\item Set $x = 4$.
\item Set $i = 3$.
\item Evaluate \lstinline{i < 5} ($i = 3$) to be \lstinline{true}.
\item Set $x = 5$.
\item Set $i = 4$.
\item Evaluate \lstinline{i < 5} ($i = 4$) to be \lstinline{true}.
\item Set $x = 6$.
\item Set $i = 5$.
\item Evaluate \lstinline{i < 5} ($i = 5$) to be \lstinline{false} and stop looping.
\end{enumerate}
In this toy example, we can see that we saved ourselves from writing one line of code; however, imagine the savings that you would have if you wanted to do \lstinline{x = x + 1} $100$ times instead.
In Listing \ref{code:loop_pseudo}, that would be as simple as replacing the $5$ with $100$.
Cool!
What would happen if we removed line $5$ from Listing \ref{code:loop_pseudo}?
In that case, the statement \lstinline{i < 5} would always be true and the loop would never stop.
This is called an \textbf{infinite loop} because the loop will run an infinite number of times betcause the terminating condition will never be false!
While many times an \textbf{infinite loop} is a bug, it is sometimes desirable behavior and maybe languages allow you to write \lstinline|LOOP { ... }| (i.e. having no \textit{terminating condition}) to explicitly declare an infinite loop.

\section{Conditionals}

\section{Functions}

\section{Structs}

\section{Exercises}

\subsection{Variables}

\begin{Exercise}

\begin{lstlisting}[caption={Pseudocode.}, label={code:exercise_var_1}, mathescape]
var x = 1
var y = x + 3
var z = y $\times$ 5
\end{lstlisting}

\begin{enumerate}
\item What is the value of \lstinline{y} after Line 2 executes (or does it not exist or is this code invalid)?
\item What is the value of \lstinline{z} after Line 2 executes (or does it not exist or is this code invalid)?
\item What is the value of \lstinline{z} after Line 3 executes (or does it not exist or is this code invalid)?
\end{enumerate}

\end{Exercise}

\begin{Exercise}

\begin{lstlisting}[caption={Pseudocode.}, label={code:exercise_var_2}]
var x = x + 5
var y = 3
\end{lstlisting}

\begin{enumerate}
\item What is the value of \lstinline{x} after Line 2 executes (or does it not exist or is this code invalid)?
\end{enumerate}

\end{Exercise}

\begin{Exercise}

\begin{lstlisting}[caption={Pseudocode.}, label={code:exercise_var_3},mathescape]
var x = 2
var y = x $\times$ 2
var z = y $\times$ 2
x = z $\times$ 4
y = x / 2
\end{lstlisting}

\begin{enumerate}
\item What is the value of \lstinline{x} after Line 2 executes (or does it not exist or is this code invalid)?
\end{enumerate}

\begin{enumerate}
\item What is the value of \lstinline{x} after Line 3 executes (or does it not exist or is this code invalid)?
\item What is the value of \lstinline{y} after Line 5 executes (or does it not exist or is this code invalid)?
\item What is the value of \lstinline{z} after Line 2 executes (or does it not exist or is this code invalid)?
\end{enumerate}

\end{Exercise}

\subsection{Loops}
\setcounter{Exercise}{0}

\begin{Exercise}
\begin{lstlisting}[caption={Pseudocode.}, label={code:exercise_loop_1},mathescape]
var x = 2
var i = 0
LOOP i $<=$ 7 {
    x = x + 5
}
\end{lstlisting}

\begin{enumerate}
\item How many times will the loop run?
\item What is the value of \lstinline{x} after the loop finishes running $3$ times.
\end{enumerate}
\end{Exercise}

\begin{Exercise}
\begin{lstlisting}[caption={Pseudocode.}, label={code:exercise_loop_2},mathescape]
var x = 1
var i = 9
LOOP i $>=$ 0 {
    x = x * 2
    i = i - 3
}
\end{lstlisting}

\begin{enumerate}
\item How many times will the loop run?
\item What is the value of \lstinline{x} after the loop finishes running.
\end{enumerate}
\end{Exercise}

\begin{Exercise}

Write a program that will
\begin{enumerate}
\item Set the value of $y$ to be $7$.
\item Increment the value of $x$ (with an initial value of your choosing) $y$ number of times.
\end{enumerate}

\end{Exercise}

\end{document}
