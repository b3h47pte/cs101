\documentclass{article}
\usepackage{amsmath}
\usepackage{fullpage}
\usepackage{listings}
\usepackage{hyperref}

\lstset{
numbers=left, 
numberstyle=\small, 
numbersep=8pt, 
frame = single, 
framexleftmargin=15pt,
basicstyle = \ttfamily,
columns=fullflexible
}

\title{Basic Building Blocks}

\begin{document}
\maketitle

\section{Overview and Preface}

This lesson will teach you the basic building blocks of programming which are
\begin{itemize}
\item Variables,
\item Loops,
\item Conditionals,
\item and Functions.
\end{itemize}
Most of computer science and software engineering is about how to effectively use the above concepts to accomplish some goal whether it be making a website like Google or running computer simulations for spaceship engines.
Note that there is a distinction between the disciplines of \textbf{computer science} and \textbf{software engineering}.
In very simplified terms, \textbf{computer science} is about finding new ways to solve some problem using a computer and \textbf{software engineering} is about finding the best way to tell the computer to solve the problem \textit{and} allowing you and your coworkers/fellow students to understand the solution weeks/months/years later.
In fact, there are many computer scientists who are horrendous software engineers and many software engineers who are terrible computer scientists \footnote{I will note that this is unideal and to be truly effective, one should know both.}.
However, they both fields share the common toolset of ``programming''.
My goal is to give you a solid foundation in programming so that you can excel in both should you choose.

Now, a common misconception about programming is that it requires one to excel in math.
However, that is simply not the case.
In fact, learning to become an effective programmer is probably more similar to learning a new language.
You certainly do not need to know multivariate calculus and partial differential equations to excel as a software engineer who creates websites and mobile applications for a living.
On the flip side, the more in depth you go in computer science, the more you will see the similarities between computer science and applied math so a strong foundation in math is definitely encouraged.

This lesson will focus on slowly introducing you to the building blocks of programming without burdening you with the harsh syntax requirements of real programming languages.
We will use an ad hoc pseudo-language to present these concepts; however, most programming languages will an equivalent so these skills should be easily transferable.
Note that I will use the word \textit{code} and \textit{program} interchangeably and I will refer to the \textit{code} written using our pseudo-language as \textit{pseudocode}.

\section{Variables}

Programming variables are very similar to the variables you learned about in Algebra.
Using your knowledge of Algebra, one can determine from Equation \ref{eq:algebra} that $y = 4$ and that $x$ and $y$ are the \textit{variables}.
\begin{equation}
\begin{aligned}
x & = 1 \\
y & = x + 3
\end{aligned}
\label{eq:algebra}
\end{equation}
Let us rewrite Equation \ref{eq:algebra} using our pseudo-language as seen in Listing \ref{code:algebra_pseudo}.
\begin{lstlisting}[caption={The equivalent of Equation \ref{eq:algebra} in our pseudo-language.}, label={code:algebra_pseudo}]
var x = 1
var y = x + 3
\end{lstlisting}
Here, we use the \lstinline{var} keyword to indicate that we are introducing a \textbf{new} \textit{variable} with the name \lstinline{x}.
The \lstinline{= 1} means the same thing as in Algebra; we are simply saying that we are assigning the \textit{variable} \lstinline{x} the value of $1$.
There is, however, one crucial difference between Algebra and programming.
\begin{equation}
\begin{aligned}
x & = 1 \\
y & = x + 3 \\
x & = x + 5
\end{aligned}
\label{eq:algebra_invalid}
\end{equation}
In math, Equation \ref{eq:algebra_invalid} makes no sense because
\begin{enumerate}
\item How can $x = 1$ while simultaneously be $x = x + 5$.
\item How can $x$ be defined using itself?
\end{enumerate}
However, the pseudocode shown in Listing \ref{code:algebra_invalid_pseudo} is perfectly valid.
\begin{lstlisting}[caption={The equivalent of Equation \ref{eq:algebra_invalid} in our pseudo-language.}, label={code:algebra_invalid_pseudo}]
var x = 1
var y = x + 3
x = x + 5
\end{lstlisting}
The first difference between code and Algebra is the concept of \textit{time}.
While Equation \ref{eq:algebra_invalid} is a singular statement of truth, Listing \ref{code:algebra_invalid_pseudo} are three separate statements that modify what is \textit{currently} true.
Thus, the proper way to read Listing \ref{code:algebra_invalid_pseudo} is in three separate steps and \textbf{each line of code gets executed in the order that they appear}.
\begin{enumerate}
\item After Line 1 executes, there is a variable \lstinline{x} with a value of $1$ and \textbf{no other variables exist}.
\item After Line 2 executes, \lstinline{x} still has a value of $1$ and there is now another variable \lstinline{y} with a value of $4$.
\item After Line 3 executes, \lstinline{x} has a value of $6$ and \lstinline{y} is unchanged.
\end{enumerate}
By convention, the right hand side of the equal sign is executed first which makes Line 3 perfectly valid.
The alternative way of writing that line is shown in Listing \ref{code:explicit_pseudo}; however, that requires two lines to write and creates a new variable and is thus less desirable.
\begin{lstlisting}[caption={A more explicit $x = x + 5$.}, label={code:explicit_pseudo}]
var z = x + 5
x = z
\end{lstlisting}
The true math equivalent of Listing \ref{code:algebra_invalid_pseudo} is thus shown in Equation \ref{eq:algebra_invalid_fix}.
\begin{equation}
\begin{aligned}
x_0 & = 1 \\
y & = x_0 + 3 \\
x & = x_0 + 5
\end{aligned}
\label{eq:algebra_invalid_fix}
\end{equation}
Finally, it is important to note that you can only set a variable equal to values that exist at that current moment in time.
Thus, writing \lstinline{var x = y + 2} in line 1 of Listing \ref{code:algebra_invalid_pseudo} would be invalid.
Similarly, writing \lstinline{var x = x + 2} in line 1 of Listing \ref{code:algebra_invalid_pseudo} would also be invalid.
Remember, the right hand side of the equal sign is executed first and at that point, \lstinline{var x} has not been executed to actually create the variable \lstinline{x}.
Thus if we just write \lstinline{var x}, then we have created the variable, the computer knows the variable \lstinline{x} exists, but \lstinline{x} has not been assigned a value.
Then if we write \lstinline{x = 5} in the next line, then we have assigned an actual value to the variable.
Before the variable is assigned a value, the value of the variable is \textbf{undefined}.

\section{Conditionals}

In our discussion of variables, every line of code gets run; however, the true power of programming comes from the fact that you can tell the computer to do different things depending on whether certain conditions are met.
For example, what if we want to have \lstinline{y} be set to a different value depending on whether \lstinline{x} is divisble by $2$, divisible by $3$, or divisible by neither.
This is where conditionals come into play and the most common conditional you will see is the \textbf{if-else if-else} conditional.

See Listing \ref{code:pseudo_conditional_1} for a stereotypical if-else if-else statement example.
In this case, after the program finishes running \lstinline{y} will be set to $3$.
\begin{lstlisting}[caption={A stereotypical IF statement.}, label={code:pseudo_conditional_1}]
var x = 10
var y
IF x % 2 == 0 {
    y = 3
}
\end{lstlisting}
In this example, we have introduced three new concepts to our pseudolanguage:
\begin{enumerate}
\item The \lstinline{IF} statement.
\item The \lstinline{==} operator.
\item The modulo operator \lstinline{%}.
\end{enumerate}
The \lstinline{IF} statement does what it sounds like it does.
\lstinline{IF} the statement (conditional) \lstinline{x % 2 == 0} is true, then execute the code within the curly braces \lstinline|{ ... }|.
Note that there can be multiple lines of code within the curly braces, commonly called a ``block'' of code.
In this case, that would be to set $y = 3$.
If the statement is not true, then ignore the code within the curly braces and move on to after the right curly brace \lstinline|}|.

Next, \lstinline{==} is the equality check operator.
That is, \lstinline{x == y} is a true statement if and only if \lstinline{x} equals \lstinline{y}.
In programming, we generally need to differentiate between a single equal sign (assigning variables) and two equal signs (equality).
Similarly, \lstinline{!=} is used to represent the inequality check operator.
That is, \lstinline{x != y} is a true statement if and only if \lstinline{x} does not equal \lstinline{y}.

Next, to determine whether something is divisble by a number you want to use the \textbf{modulo} operation, commonly denoted by the \lstinline{%} symbol.
If you write \lstinline{z = x % y}, then \lstinline{z} will equal the remainder after you divide \lstinline{x} by \lstinline{y}.
Thus, to determine is a number \lstinline{x} is divisible by another number \lstinline{y}, all you need to check is if \lstinline{x} modulo \lstinline{y} is equal to $0$.

Now, let us introduce the \lstinline{ELSE IF} and \lstinline{ELSE} statements that are commonly used in conjunction with the \lstinline{IF} statement.
\begin{lstlisting}[caption={A stereotypical IF-ELSE IF-ELSE statement.}, label={code:pseudo_conditional_2}]
var x = 10
var y
IF x % 3 == 0 {
    y = 3
} ELSE IF x % 5 == 0 {
    y = 42
} ELSE {
    y = 27
}
\end{lstlisting}
In Listing \ref{code:pseudo_conditional_2}, we see that the initial \lstinline{IF} statement has a conditional that is false because $10$ is not divisible by $3$.
Then, the code will move on to the \lstinline{ELSE IF} statement to check if \lstinline{x} is divisible by $5$.
In this example, it is so \lstinline{y} is set to $42$.
However, if \lstinline{x} was not divisible by $5$ then the catch-all \lstinline{ELSE} statement would evaluate and thus the code would set \lstinline{y} to $27$.
Note that you can not have an \lstinline{ELSE IF} or an \lstinline{ELSE} without the initial \lstinline{IF} statement.
Furthermore, you can only have one \lstinline{ELSE} statement for every \lstinline{IF} statement.
You can, however, have any number of \lstinline{ELSE IF} statements.
One helpful way of reading code in general is to translate the program into plain English.
In this case, this program would read
\begin{lstlisting}
First, set x to be equal to 10.
Then create the variable y but do not assign it a value.
Then if x is divisible by 3, set y to be 3.
If x is not divisible by 3 but is divisible by 5, set y to be 42.
If x is neither by 3 nor 5, then set y to be 27.
\end{lstlisting}

Note, however, that there is a subtle but important difference between using an \lstinline{IF-ELSE IF} statement and using two separate \lstinline{IF} statements.
Compare the code in Listing \ref{code:pseudo_if_else_if} versus the code in Listing \ref{code:pseudo_if_if}.

\begin{lstlisting}[caption={IF ELSE-IF.}, label={code:pseudo_if_else_if}]
var x = 10
var y
IF x % 2 == 0 {
    y = 2
} ELSE IF x % 5 == 0 {
    y = 5
}
\end{lstlisting}

\begin{lstlisting}[caption={IF IF.}, label={code:pseudo_if_if}]
var x = 10
var y
IF x % 2 == 0 {
    y = 2
} 

IF x % 5 == 0 {
    y = 5
}
\end{lstlisting}
After the program in Listing \ref{code:pseudo_if_else_if} runs, the value of \lstinline{y} is $2$ because the \lstinline{ELSE IF} statement \textbf{only} runs if the initial \lstinline{IF} statement's conditional evaluates to be false.
Compare that to the program in Listing \ref{code:pseudo_if_if} where the value of \lstinline{y} is $5$ because both conditionals evaluate to true and therefore the code within both \lstinline{IF} statements are run.
Therefore, the line of code that is run last is the one that sets the value of \lstinline{y}.

\section{Loops}

Sometimes one has the need to do something multiple times.
Take Listing \ref{code:loop_pseudo_unrolled} for example.
While it is somewhat contrived as we could just set \lstinline{var x = 6} to begin with, let us roll with it.
\begin{lstlisting}[caption={Unless you know what are you doing (you don't), don't do this.}, label={code:loop_pseudo_unrolled}]
var x = 1
x = x + 1
x = x + 1
x = x + 1
x = x + 1
x = x + 1
\end{lstlisting}
Effectively, this code is saying we want to increment \lstinline{x} (\lstinline{x = x + 1}) $5$ times.
Thus instead of writing lines $2$ through $6$ and duplicating code (generally bad!), we could write a loop instead as seen in Listing \ref{code:loop_pseudo}.
\begin{lstlisting}[caption={Loops are cool.}, label={code:loop_pseudo}]
var x = 1
var i = 0
LOOP i < 5 {
    x = x + 1
    i = i + 1
}
\end{lstlisting}
Here, the statement \lstinline|LOOP i < 5 { ... } | means that we want to \textbf{loop} (repeat) everything between the curly braces (\lstinline|{ ... }|) while the \textit{terminating condition} \lstinline{i < 5} is true.
Thus, Listing \ref{code:loop_pseudo} can be interpreted as
\begin{enumerate}
\item Create a new variable \lstinline{x} and set it to $1$.
\item Create a new variable \lstinline{i} and set it to $0$. \footnote{By convention, \textit{most} programming languages start counting at $0$, get used to it!}
\item Evaluate \lstinline{i < 5} ($i = 0$) to be \lstinline{true}.
\item Set $x = 2$.
\item Set $i = 1$.
\item Evaluate \lstinline{i < 5} ($i = 1$) to be \lstinline{true}.
\item Set $x = 3$.
\item Set $i = 2$.
\item Evaluate \lstinline{i < 5} ($i = 2$) to be \lstinline{true}.
\item Set $x = 4$.
\item Set $i = 3$.
\item Evaluate \lstinline{i < 5} ($i = 3$) to be \lstinline{true}.
\item Set $x = 5$.
\item Set $i = 4$.
\item Evaluate \lstinline{i < 5} ($i = 4$) to be \lstinline{true}.
\item Set $x = 6$.
\item Set $i = 5$.
\item Evaluate \lstinline{i < 5} ($i = 5$) to be \lstinline{false} and stop looping.
\end{enumerate}
In this toy example, we can see that we saved ourselves from writing one line of code; however, imagine the savings that you would have if you wanted to do \lstinline{x = x + 1} $100$ times instead.
In Listing \ref{code:loop_pseudo}, that would be as simple as replacing the $5$ with $100$.
Cool!
What would happen if we removed line $5$ from Listing \ref{code:loop_pseudo}?
In that case, the statement \lstinline{i < 5} would always be true and the loop would never stop.
This is called an \textbf{infinite loop} because the loop will run an infinite number of times betcause the terminating condition will never be false!
While an \textbf{infinite loop} is oftentimes a bug \footnote{A bug means your programming is behaving in a manner that is not what you expected/want.}, it is sometimes desirable behavior and some languages allow you to write \lstinline|LOOP { ... }| (i.e. having no \textit{terminating condition}) to explicitly create an infinite loop.

\section{Functions}

From Algebra, you should know that a function $f(x)$ takes in a variable $x$ and computes a result $y$.
Functions in programming do (almost) the same thing!
They can also take in some input(s) to compute some output(s); however, they don't necessarily have to have any inputs or outputs.
Let us create a function based off the code from Listing \ref{code:loop_pseudo}

\begin{lstlisting}[caption={Your first function.}, label={code:fn_pseudo}]
FUNCTION test(var x) {
    var i = 0
    LOOP i < 5 {
        x = x + 1
        i = i + 1
    }
    return x
}
\end{lstlisting}

Here, we created a new \lstinline{FUNCTION} called \lstinline{test} that takes in a variable that we call \lstinline{x} as input.
The \lstinline{FUNCTION test(var x)} part is oftentimes called the \textbf{function signature}.
Every time we ``call'' the \lstinline{test} function, all the code between the curly brackets will be run \lstinline|{ ... }|.
This function will compute a result that gets \lstinline{returned} to whoever called the function.
If there was no \lstinline{return} statement then this function would have zero outputs.
Thus we can write
\begin{lstlisting}[caption={Call test.}, label={code:call_fn_pseudo}]
var z = 5
var x = 2
var w = test(z)
var a = test(1)
var b = test(x)
\end{lstlisting}
to get $w = 11$ and $a = 6$ and $b = 7$ by ``calling'' the \lstinline{test} function three times with different inputs each time.

\subsection{Variable Scope}

Before this point, every time we wrote \lstinline{var x}, we assumed that the variable \lstinline{x} would exist from the point where you saw \lstinline{var x} to when the program ends.
However, now we see that we wrote \lstinline{var x} twice: once in Listing \ref{code:fn_pseudo} and once in Listing \ref{code:call_fn_pseudo}.
What do?
This is where the concept of \textbf{variable scopes} come into play.
In other words, the \lstinline{var x} inside the function has nothing to do with the \lstinline{var x} outside of the function, they are two separate variables!
Every time you call the \lstinline{test} function, your computer will create a new variable \lstinline{x} that is: 1) separate from the \lstinline{var x} that is outside of the function and 2) separate from the \lstinline{var x} from the last time you called the function.
The same applies for the variable \lstinline{i}.
Note that the specific rules of variable scopes depend on the programming language so we will go into more detail about variable scopes once we start dealing with a real programming language.

\subsection{Function Input(s)}

You may have noticed that in Listing \ref{code:fn_pseudo} we only wrote \lstinline{var x} without assigning \lstinline{x} a value.
You would be correct; however, note that \lstinline{x} is not \textbf{undefined} but rather the computer will implicitly set \lstinline{x} to whatever value gets passed in as the input of the function.
Thus, if you write \lstinline{test(27)} then inside the \lstinline{test} function, \lstinline{x} gets set to $27$.

You can also write functions that take in multiple inputs.
In that case, you would just add additional variable inputs within the parentheses in the function signature.
For example: \lstinline{FUNCTION test_2(var x, var y, var z)} would be a function that takes in $3$ inputs.
You would call this function in the same way, i.e. \lstinline{test_2(100, 24, -34)}.

\subsection{Function Output}

The function's output is determined by the value coming after the \lstinline{return} statement.
In this case, we just wrote the variable \lstinline{x}, but we could also write a statement like \lstinline{x + 2}.
When/if the \lstinline{return} statement is run, the function immediately ends regardless of whether or not there are lines of code that would have otherwise run if the \lstinline{return} statement was not there.
Thus, the function in Listing \ref{code:fn_unreachable} will always return $20$.
\begin{lstlisting}[caption={Unreachable Code.}, label={code:fn_unreachable}]
FUNCTION test() {
    return 20
    var x = 5
    return x
}
\end{lstlisting}

\end{document}
