\documentclass{article}
\usepackage{amsmath}
\usepackage{fullpage}
\usepackage{listings}
\usepackage{exercise}

\lstset{
numbers=left, 
numberstyle=\small, 
numbersep=8pt, 
frame = single, 
framexleftmargin=15pt,
basicstyle = \ttfamily,
columns=fullflexible
}

\title{Numbers: Exercises}

\begin{document}
\maketitle

\section{Number Conversion}

\begin{Exercise}

Write the following decimal numbers in both binary and hexadecimal form:
\begin{enumerate}
\item 255
\item 64
\item 45
\item 15
\item 33
\end{enumerate}

\end{Exercise}

\begin{Exercise}

Write the following numbers in decimal form.
If the number is given in binary form, write its hexadecimal equivalent.
If the number is given in hexadecimal form, write its binary equivalent.
\begin{enumerate}
\item \lstinline{0b1111}
\item \lstinline{0xFA}
\item \lstinline{0b1001}
\item \lstinline{0x91}
\item \lstinline{0b0010}
\end{enumerate}

\end{Exercise}

\section{Bits and Bytes}

\begin{Exercise}

What is the largest number representable by the following number of bits (you can write the number in power form):

\begin{enumerate}
\item 8
\item 3
\item 32
\item 64
\end{enumerate}

\end{Exercise}

\begin{Exercise}

How many different numbers are representable by the following number of bits (you can write the number in power form):

\begin{enumerate}
\item 8
\item 3
\item 32
\item 64
\end{enumerate}

\end{Exercise}

\begin{Exercise}

You buy a new, \textit{super fancy}, laptop off Amazon in preparation for college that has $16$ gigabytes of RAM.
How many $64$ bit numbers can it store?

\end{Exercise}

\end{document}
